
\lecture{1}

\section{Введение}
\subsection{Предисловие}
Ну чето рассказали там про принцип устойчивости частот, про то, что ля-ля-ля тополя

\subsection{Напоминание с ОВиТМа}

\begin{definition}
    \(\mathcal{F}\) --- алгебра над \(\Omega\), если 
    \begin{enumerate}
        \item \(\Omega \in \mathcal{F}\)
        \item \(A, B \in \mathcal{F} \Ra A \cap B \in \mathcal{F}\)
        \item \(A \in \mathcal{F} \Ra \overline{A} \in \mathcal{F}\)
    \end{enumerate}
\end{definition}

\begin{note}
    Элементы \(\Omega\) называются элементарными событиями.
\end{note}

\begin{note}
    Элементы \(\mathcal{F}\) называются событиями.
\end{note}

\begin{definition}
    \(\mathcal{F}\) --- \(\sigma\)-алгебра над \(\Omega\), если 
    \begin{enumerate}
        \item \(\mathcal{F}\) --- алгебра
        \item \(A_1, \dots \in \mathcal{F} \Ra \bigcup_{i = 1}^\infty A_i \in \mathcal{F}\)
    \end{enumerate}
\end{definition}

\begin{definition}
    Функция \(P: \mathcal{F} \ra [0, 1]\) называется вероятностной мерой, если \(P(\Omega) = 1, P\) --- \(\sigma\)-аддитивна.
\end{definition}

\begin{note}
    \begin{enumerate}
        \item \(P(\emptyset) = 0\)
        \item \(A \subset B \Ra P(A) \le P(B)\) --- \textbf{монотонность меры}
        \item \(P\) конечно аддитивна
        \item Для \(P\) верна формула включений-исключений.
        \item \(P\left( \bigcup_{i = 1}^\infty A_i \right) = \le \sum_{i = 1}^\infty P(A_i)\)
    \end{enumerate}
\end{note}

\begin{theorem}[О непрерывности вероятностной меры]
    Пусть \((\Omega, \mathcal{F})\) таково, что \(\mathcal{F}\) --- алгебра над \(\Omega\), \(P\) --- мера и \(P(\Omega) = 1\). Тогда следующие условия эквивалентны:
    \begin{enumerate}
        \item \(P\) --- \(\sigma\)-аддитивна на \(\mathcal{F}\)
        \item \(P\) непрерывна в нуле, т.е. \(\bigcap A_i = \emptyset \Ra P(A_i) \ra P(A)\)
        \item \(P\) непрерывна сверху, т.е. \(\bigcap A_i = A \Ra P(A_i) \ra P(A)\)
        \item \(P\) непрерывна снизу, т.е. \(\bigcup A_i = A \Ra P(A_i) \ra P(A)\)
    \end{enumerate}
\end{theorem}

\begin{definition}
    Вероятностное пространство --- это измеримое пространство \((\Omega, \mathcal{F}, P)\), т.е. \(\Omega\) --- множество, \(\mathcal{F}\) --- некая \(\sigma\)-алгебра над \(\Omega\), \(P\) --- вероятностная мера на \((\Omega, \mathcal{F})\).
\end{definition}

\begin{definition}
    Система \(\mathcal{M}\) подмножеств \(\Omega\) называется \(\pi\)-системой, если \(\mathcal{M}\) замкнуто относительно счетного пересечения
\end{definition}

\begin{definition}
    Система \(\mathcal{M}\) подмножеств \(\Omega\) называется \(\lambda\)-системой, если
    \begin{enumerate}
        \item \(\Omega \in \mathcal{M}\)
        \item \(A, B \in \mathcal{M}, A \subset B \Ra B \setminus A \in \mathcal{M}\)
        \item \(A_i \in \mathcal{M} \Ra \bigcup A_i \in \mathcal{M}\)
    \end{enumerate}
\end{definition}

\begin{theorem}[Первая теорема о \(\lambda\)-системах]
    Система \(\mathcal{F}\) является \(\sigma\)-алгеброй над \(\Omega\) тогда и только тогда, когда она является \(\lambda\)-системой и \(\pi\)-системой.
\end{theorem}

\begin{proposition}
    Для любой системы \(\mathcal{M}\) подмножеств \(\Omega\) существует минимальная по включению, содержащаяся в \(\mathcal{M}\)
\end{proposition}

\begin{note}
    \(\sigma(\mathcal{M}), \alpha(\mathcal{M}), \pi(\mathcal{M}), \lambda(\mathcal{M})\) --- порожденные (минимальные) \(\sigma\)-алгебра, алгебра, \(\pi\)-система, \(\lambda\)-система соответственно.
\end{note}

\begin{theorem}[Вторая теорема о \(\lambda\)-системах]
    Если \(\mathcal{M}\) --- \(\pi\)-система на \(\Omega\), то \(\sigma(\mathcal{M}) = \lambda(\mathcal{M})\)
\end{theorem}
\begin{proof}
    См. доказательство из курса ОВиТМа
\end{proof}

\begin{example}[Борелевская \(\sigma\)-алгебра]
    \(\mathcal{B}(\R)\) --- \(\sigma\)-алгебра, порожденная всеми открытыми множествами (или, что равносильно, всеми полуинтервалами)
\end{example}

\begin{example}[Борелевская \(\sigma\)-алгебра в \(\R^n\)]
    \(\mathcal{B}(\R^n)\) --- \(\sigma\)-алгебра, порожденная всеми открытыми множествами (или, что равносильно, всеми кубами, где куб --- декартово произведение борелевских множеств).
\end{example}

\begin{example}
    \(\mathcal{B}(\R^\infty)\) --- \(\sigma\)-алгебра, порожденная всеми цилиндрами. Цилиндр --- множество \({x \in \R^\infty : (x_1, \dots x_n) \in B}, B \in \mathcal{B}(\R^n)\)
\end{example}


\begin{example}[Борелевская \(\sigma\)-алгебра в общем случае]
    Пусть \((S, \rho)\) --- метрическое пространство, тогда \(\mathcal{B}(S)\) --- \(\sigma\)-алгебра, порожденная всеми открытыми множествами.
\end{example}

\section{Вероятностная мера на прямой}
Пусть \(P\) --- вероятностная мера на \(\R, \mathcal{B}(\R)\)
\begin{definition}
    Функцией распределения называется функция \(F(x) = P((-\infty, x]), x \in \R\).
\end{definition}

\begin{lemma}[О свойствах функции распределения]
    \indent
    \begin{enumerate}
        \item \(F\) не убывает
        \item \(\lim_{x \ra -\infty} F(x) = 0\), \(\lim_{x \ra +\infty} F(x) = 1\)
        \item \(F\) непрерывна справа
    \end{enumerate}
\end{lemma}
\begin{proof}
    \indent
    \begin{enumerate}
        \item \(x \le y \Ra (-\infty, x] \subset (-\infty, y] \Ra F(x) \le F(y)\)
        \item \(x_n \ra -\infty \Ra (-\infty, x_n] \ra \emptyset \Ra \lim_{n \ra \infty} F(x_n) = 0 = \lim_{x \ra +\infty} F(x)\), \(x_n \ra +\infty \Ra (-\infty, x_n] \ra \R \Ra \lim_{n \ra \infty} F(x_n) = 1 = \lim_{x \ra +\infty} F(x)\)
        \item Если \(x_n \searrow x\), то \((-\infty, x_n] \searrow (-\infty, x]\) и \(F(x_n) \ra F(x)\)
    \end{enumerate}
\end{proof}

\begin{theorem}[О взаимно-однозначном соответствии функции распределения и вероятностной меры]
    Пусть \(F\) удовлетворяет условиям 1-3 из предыдущей теоремы. Тогда существует единственная вероятностная мера \(P\) на \((\R, \mathcal{B}(\R))\), т.ч. \(F\) --- её функция распределения.
\end{theorem}

\subsection{Классификация вероятностных мер}

Далее мы будем отождествлять понятия распределения и вероятностной меры.

\subsubsection{Дискретные распределения}

\begin{definition}
    Вероятностная мера \(P\) на \((R, \mathcal{B}(R))\) называется дискретной, если \(\exists X\) --- не более. чем счетое множество на \(\R\), такое, что \(P(\R \setminus X)\) и \(\forall x \in X P(\{x\}) > 0\).
\end{definition}

\paragraph{Примеры}
\begin{enumerate}
    \item Распределение Бернулли: \(P \sim Bern(p)\), если \(X = \{0, 1\}, P(\{1\}) = p, P(\{0\}) = 1 - p\)
    \item Равномерное распределение: \(X = {1, 2, \dots n}\), \(P(\{i\}) = \frac{1}{n}\)
    \item Биномиальное распределение: \(P \sim Bin(n, p)\), если \(X = \N, P(\{k\}) = C_n^kp^k(1 - p)^{n - k}, P(\{0\})\), моделирует количество успехов среди \(n\) испытаний.
    \item Пуассоновское распределение: \(P \sim Pois(n, p)\), если \(X = \N, P(\{k\}) = \frac{\lambda^k}{k!}e^{-\lambda}, P(\{0\})\), моделирует редкие события
    \item Геометрическое распределение: \(P \sim Geom(p)\), если \(X = \N, P(\{k\}) = p(1 - p)^{k-1}, P(\{0\})\), моделирует первый момент удачи в бесконечной схеме испытаний Бернулли
\end{enumerate}

\subsubsection{Абсолютно непрерывные распределения}

\begin{definition}
    \(F(x)\) называется абсолютно непрерывной, если \(\exists p(t) \ge 0: \forall x \in \R F(x) = \int_{-\infty}^x p(t)dt\). В таком случае мы говорим, что \(p(t)\) является плотностью функции \(F\) или соответствующего распределения (вероятностной меры).
\end{definition}

\begin{note}
    В таком случае \(F'(x) = p(x)\) почти всюду по мере Лебега.
\end{note}

\paragraph{Примеры}
\begin{enumerate}
    \item Равномерное распределение на отрезке \([a, b]\) --- \(U(a, b)\): 
    \[p(x) = \left\{\begin{array}{l}
        \frac{1}{b - a}, a \le x \le b \\
        0, \text{иначе}
    \end{array}\right., F(x) = \left\{\begin{array}{l}
        0, x < a \\
        \frac{x - a}{b - a}, a \le x \le b \\
        1, x > b
    \end{array}\right.\]
    Моделирует случайную точку из отрезка \([a, b]\)
    \item Нормальное распределение --- \(N(a, \sigma^2)\):
    \[p(x) = \frac{1}{\sqrt{2\pi \sigma^2s}}e^{-\frac{(x - a)^2}{2\sigma^2}}\]
    Моделирует измерение с ошибкой

    \item Экспоненциальное распределение --- \(N(a, \sigma^2)\):
    \[p(x) = \lambda e^{\lambda x} I\{x > 0\}\]
    \[F(x) = \left\{\begin{array}{l}
        0, x < 0 \\
        1 - e^{-\lambda x}, x \ge 0
    \end{array}\right.\]
    Моделирует время ожидание

    \item Гамма-распределение: \(\Gamma(\lambda, \alpha), \lambda, \alpha > 0\).
    \[p(x) = \frac{x^{\alpha - 1}\lambda^\alpha}{\Gamma(\alpha)}e^{-\lambda x}I\{x > 0\}\]
    Где:
    \[\Gamma(\alpha) = \int_0^{+\infty} x^{\alpha - 1}e^{-x}dx\]
    Нам в дальнейшем потребуются различные свойства \(\Gamma\)-функции: \(\Gamma(n) = (n - 1)!, \Gamma(\alpha + 1) = \alpha\Gamma(\alpha)\)

    \item Распределение Коши \(K(\sigma), \sigma > 0\)
    \[p(x) = \frac{\sigma}{\pi(x^2 + \sigma^2)}\]
    \[F(x) = \frac{1}{2} + \frac{1}{\pi}\arctan \frac{x}{\sigma}\]
    Модель
\end{enumerate}

\lecture{2}

\subsubsection{Сингулярные распределения}

\begin{definition}
    Точка \(x\) называется точкой роста фукнции распределения \(F(x)\), если \(\forall \epsilon > 0: F(x + \epsilon) - F(x - \epsilon) > 0\)
\end{definition}

\begin{definition}
    Функция распределения \(F(x)\) наызвается сингулярной, если она непрерывна, и множество точек ее роста имеет \(\mu = 0\).
\end{definition}

\paragraph{Примеры}
\begin{enumerate}
    \item Канторова лестница --- ее точками роста ялвяется канторово множество.
\end{enumerate}

\begin{theorem}[Лебега о разложении]
    Если \(F(x)\) --- функция распределения на прямой, тогда \(\exists !\) разложение вида: \(F(x) = \alpha_1F_1(x) + \alpha_2F_2(x) + \alpha_3F_3(x)\), где \(\alpha_i \ge 0, \alpha_1 + \alpha_2 + \alpha_3 = 1\) и \(F_1\) --- дискретная, \(F_2\) --- абсолютно непрерывная, \(F_3\) --- сингулярная.
\end{theorem}

\section{Вероятностные меры в \(\R^n\)}
Пусть \(P\) --- вероятностная мера на \((\R^n, \mathcal{B}(\R^n))\). 
\begin{definition}
    Функцией распределения \(P\) называется \(F(x_1, \dots x_n) = P((-\infty, x_1] \times \dots \times (-\infty, x_n])\)
\end{definition}

\begin{note}[Обозначения]
    \begin{enumerate}
        \item \(\vec{x} = (x_1, \dots x_n)\)
        \item \(\vec{x} \ge \vec{y} \Lra x_i \ge y_i\)
        \item \((-\infty, \vec{x}] = (-\infty, x_1] \times \dots \times (-\infty, x_n]\)
        \item \(\vec{x}_n \downarrow \vec{x}\) если \(\vec{x} = \lim_{n \ra \infty}\vec{x}_n\) и \(\vec{x}_n \ge \vec{x}_{n + 1}\).
    \end{enumerate}
\end{note}

\begin{definition}
    Для \(i = 1, \dots n, a_i < b_i\) введем:
    \[\Delta_{a_i, b_i}^iF(x_1, \dots x_n) = F(x_1, x_{i - 1}, b_i, x_{i + 1}, \dots x_n) - F(x_1, x_{i - 1}, a_i, x_{i + 1}, \dots x_n)\]
\end{definition}

\begin{lemma}[Свойства многомерных функций распределения]
    \begin{enumerate}
        \item Если \(\vec{x}_n \downarrow \vec{x}\), то \(F(\vec{x}_n) \ra F(\vec{x})\)
        \item \(\lim_{\begin{array}{c}
            x_1 \ra \infty \\
            \vdots \\
            x_n \ra \infty
        \end{array}} F(x_1, \dots x_n) = 1\)
        \item \(\forall i = 1, \dots n: \lim_{x_i \ra -\infty} F(x_1, \dots x_n) = 0\).
        \item Для любых \(a_i < b_i, i = 1, \dots n\):
        \[\Delta_{a_1, b_1}^1 \circ \Delta_{a_2, b_2}^2 \circ \dots \circ \Delta_{a_n, b_n}^n F(x_1, \dots x_n) \ge 0\]
    \end{enumerate}
\end{lemma}
\begin{proof}\indent
    \begin{enumerate}
        \item Если \(\vec{x}_n \downarrow \vec{x}\), то \((-\infty, \vec{x}_n] \downarrow (-\infty, \vec{x}] \Ra\) в силу непрерывности вероятностной меры, получаем \(F(\vec{x}_n) \ra F(\vec{x})\)
        \item Если \(\vec{x}_n \uparrow (+\infty, \dots +\infty)\), то \((-\infty, \vec{x}_n] \uparrow \R^n \Ra\) в силу непрерывности вероятностной меры, получаем \(F(\vec{x}_n) \ra P(\R^n) = 1\)
        \item Если \(x_i \downarrow -\infty\), то \((-\infty, \vec{x}_n] \downarrow \emptyset \Ra\) в силу непрерывности вероятностной меры, получаем \(F(\vec{x}_n) \ra P(\emptyset) = 0\)
        \item Для \(n = 2\):
        \[\Delta^1_{a_1, b_1} \circ \Delta^2_{a_2, b_2} F(x_1, x_2) = F(b_1, b_2) - F(b_1, a_2) - F(a_1, b_2) + F(a_1, a_2) = P((a_1, b_1] \times (a_2, b_2]) \ge 0\]
        В общем случае:
        \[\Delta_{a_i, b_i}^i P(A_1 \times \dots \times A_{i - 1} \times (-\infty, x_i) \times A_{i + 1} \times \dots \times A_n) = \]
        \[= P(A_1 \times \dots \times A_{i - 1} \times (a_i, b_i] \times A_{i + 1} \times \dots \times A_n)\]
        Получаем, что 
        \[\Delta_{a_1, b_1}^1 \circ \Delta_{a_2, b_2}^2 \circ \dots \circ \Delta_{a_n, b_n}^n F(x_1, \dots x_n) = P((a_1, b_1] \times \dots \times (a_n, b_n]) \ge 0\]
    \end{enumerate}
\end{proof}

\begin{theorem}[О взаимно однозначном соответствии]
    Если \(F\) удовлетворяет свойствами 1-3 из леммы, то \(\exists !\) вероятностная мера \(P\) на \((\R^n, \mathcal{B}(\R^n))\), такая, что \(F\) --- ее функция распределения
\end{theorem}
\begin{proof}
    См. ОВиТМ
\end{proof}

\begin{note}
    Свойство 3 нельзя заменить на неубывание по каждой из переменнх.
\end{note}
\begin{proof}
    Рассмотрим \(F(x, y) = \left\{\begin{array}{l}
        1, x + y \ge 0 \\
        0, x + y < 0
    \end{array}\right.\).
    Заметим, что \(F\) удовлетворяет свойствам 1, 2 и не убывает по обеим переменным. При этом, если мы возьем:
    \[\Delta^x_{-1, 1} \circ \Delta^y_{-1, 1} F(x, y) = F(1, 1) - F(-1, 1) - F(1, -1) + F(-1, -1) = 1 - 2 + 0 < 0\]
    Получаем, что \(F\) --- не двумерная функция распределения.
\end{proof}

\subsection{Примеры}
\begin{enumerate}
    \item Пусть \(F_1, F_2, \dots F_n\) --- одномерные функции распределения. Рассмотрим \(F(x_1, \dots x_n) = F_1(x_1)\dots F_n(x_n)\) --- многомерная функция распределения. Свойства 1, 2 очевидны, проверим 3:
    \[\Delta_{a_1, b_1}^1 \circ \Delta_{a_2, b_2}^2 \circ \dots \circ \Delta_{a_n, b_n}^n F(x_1, \dots x_n) = \prod_{k = 1}^n (F_k(b_k) - F_k(a_k)) \ge 0\]

    \item Пусть \(p(t_1, \dots t_n) \ge 0\), т.ч.
    \[\int_{\R^n} p(t_1, \dots t_n) dt_1dt_2\dots dt_n = 1\]
    Тогда:
    \[(*)\;\;F(x_1, \dots x_n) = \int_{-\infty}^{x_1}\int_{-\infty}^{x_2}\dots \int_{-\infty}^{x_n} p(t_1, \dots t_n)dt_1dt_2\dots dt_n\]
    Свойства 1, 2 очевидны, проверим 3:
    \[\Delta_{a_1, b_1}^1 \circ \Delta_{a_2, b_2}^2 \circ \dots \circ \Delta_{a_n, b_n}^n F(x_1, \dots x_n) = \int_{a_1}^{b_1}\int_{a_2}^{b_2}\dots \int_{a_n}^{b_n} p(t_1, \dots t_n)dt_1dt_2\dots dt_n\]
\end{enumerate}

\begin{definition}
    Если имеет место представление \((*)\), то \(p(t_1, \dots t_n)\) называется плотностью функции распределения \(F\).
\end{definition}

\section{Вероятностные меры в \(\R^\infty\)}

\begin{definition}
    Пусть \(P\) --- вероятностная мера на \((\R^{\infty}, \mathcal{B}(\R^\infty))\). Рассмотрим для \(n \in \N\) вероятностную меру \(P_n\) на \((\R^n, \mathcal{B}(\R^n))\), т.ч.
    \[P_n(B) = P(Cyl(n, B)), B \in \mathcal{B}(\R^n)\]
    Тогда можно заметить, что \(P_{n + 1}(B \times \R) = P_n(B)\).
\end{definition}

\begin{definition}
    Свойство выше называется согласованностью для последовательности вероятностных мер \(\{P_n\}\)
\end{definition}

\begin{theorem}[Колмогорова, о мерах в \(\R^\infty\)]
    Пусть \(\{P_n, n \in \N\}\) --- последовательность согласованных вероятностных мер, \(P_n\) --- мера на \(\R^n\). Тогда \(\exists !\) вероятностная мера \(P\) на \((\R^\infty, \mathcal{B}(R^\infty))\), такая, что \(\forall n \in \N: \forall B_n \in \mathcal{B}(\R^n)\):
    \[(*)\;\;\;P_n(B_n) = P(Cyl(n, B_n))\]
\end{theorem}
\begin{proof}
    Зададим меру \(P\) на цилиндрах по правилу \((*)\). Цилиндры образуют алгебру \(\mathcal{A}\). Проверим корректность задания \(P\). Если \(Cyl(n, B_n) = Cyl(n + k, B_{n + k})\), то \(B_{n + k} = B_n \times \R^k\). Тогда в силу согласованности:
    \[P_n(B_n) = P_{n + k}(B_{n + k})\]
    Проверим, что \(P\) --- конечно аддитивна на \(\mathcal{A}\). Если \(\tilde{B_1}, \dots \tilde{B_N} \in \mathcal{A}\) --- непересекаются, то будем считать, что \(\exists n: \tilde{B_i} = Cyl(n, B_i), i = 1, \dots N, B_i \in \mathcal{B}(\R^n)\). Тогда:
    \[P\left( \bigsqcup_{i = 1}^N \tilde{B_i} \right) = P\left( Cyl\left( n, \bigsqcup_{i = 1}^N \tilde{B_i} \right) \right) = P_n\left( \bigsqcup_{i = 1}^N B_i \right) = \sum_{i = 1}^N P_n(B_i) = \sum_{i = 1}^N P(\tilde{B_i})\]

    Проверим, что \(P\) непрерывна в нуле (на \(\mathcal{A}\)). Пусть \(\tilde{B_n} \downarrow \emptyset, \tilde{B_n} \in \mathcal{A}\). Без ограничения общности, считаем, что \(\tilde{B_n} = Cyl(n, B_n)\). 

    От противного. Пусть \(\lim_{n \ra \infty} P(\tilde{B_n}) = \delta > 0\). Тогда для \(\forall n \in \N\) выберем компактные \(A_n \subset \R^n\), такие, что \(A_n \subset B_n\) и \(P(B_n \setminus A_n) \le \frac{\delta}{2^{n + 1}}\). Обозначим \(\tilde{A_n} = Cyl(n, A_n)\). Тогда: \(P(\tilde{B_n} \setminus \tilde{A_n}) = P_n(B_n \setminus A_n) \le \frac{\delta}{2^{n + 1}}\). Введем \(\tilde{C_n} = \bigcap_{i = 1}^n \tilde{A_i}\). Тогда \(\tilde{C_n} \downarrow \emptyset, \tilde{C_n} = Cyl(n, C_n)\), где \(C_n = A_n \cap (A_{n - 1} \times \R) \cap (A_{n - 2} \times \R^2) \cap \dots \cap (A_{1} \times \R^{n - 1})\) --- тоже компакт в \(\R^n\). Далее:
    \[P(\tilde{B_n} \setminus \tilde{C_n}) \le \sum_{i = 1}^n P(\tilde{B_n} \setminus \tilde{A_i}) \le |B_i \subset B_n, i \ge n| \le \sum_{i = 1}^n P(\tilde{B_i} \setminus \tilde{A_i}) \le \frac{\delta}{2} \Ra \lim_{n \ra \infty} P(\tilde{C_n}) \ge \frac{delta}{2} > 0\]
    Возьмем в каждом \(\tilde{C_n}\) по точке \((x_1^{(n)}, x_2^{(n)}, \dots) \in \tilde{C_n}\). Тогда \((x_1^{(n)}, \dots x_n^{(n)}) \in C_n\). Рассмотирм последовательность \(x_1^{(n)}\). Эти все точки лежат в \(C_1\). Выберем подпоследовательность \(n_{k_i}\)  \(x_1^{(n_{k_i})} \ra x_1^0 \in C_1\). Для \(k \in \N\) построим последовательности \((x_1^{(n_{k_i})}, x_2^{(n_{k_i})}, \dots x_k^{(n_{k_i})})\), которая сходится к \(x^{(0)}_k, \dots x^{(0)}_k\) в \(С_k\). Возьмем диагональ, т.е положим \(m_k = n_{k_k}\). Тогда \(\forall s \in \N: (x_1^{(m_k)}, \dots x_s^{(m_k)}) \ra (x_1^{(0)}, \dots x_s^{(0)})= x^{(0)}\). Получаем, что \(P\) счетно-аддитивна на \(\mathcal{A}\). В таком случае по теореме о продолжении меры, \(P\) единственным образом продолжается до вероятностной меры на \(\sigma(\mathcal{A}) = \mathcal{B}(\R^\infty)\)
\end{proof}

\lecture{3}

\section{Случайные величины и векторы}
Пусть \((\Omega, \mathcal{F}, P)\) --- вероятностное пространство, \((E, \mathcal{E})\) --- измеримое пространство

\begin{definition}
    Отображение \(X: \Omega \ra E\) называется случайным элементом, если \(X\) --- измеримо, т.е. \(\forall B \in \mathcal{E}: X^{-1}(B) = {\omega: X(\omega) \in B} \in \mathcal{F}\)
\end{definition}

\begin{definition}
    Если \((E, \mathcal{E}) = (\R, \mathcal{B}(\R))\), то \(X\) называется случайной величиной.
\end{definition}

\begin{definition}
    Если \((E, \mathcal{E}) = (\R^n, \mathcal{B}(\R^n))\), то \(X\) называется случайным вектором.
\end{definition}

\begin{reminder}[Критерий измеримости отображения]
    Если \(\mathcal{M} \subset \mathcal{E}\) и \(\sigma(\mathcal{M}) = \mathcal{E}\), то \(X\) --- случайный элемент \(\Lra \forall B \in \mathcal{M} X^{-1}(B) \in \mathcal{F}\)
\end{reminder}

\begin{corollary}[Эквивалентные определения случайных величин и векторов]
    \begin{enumerate}
        \item \(\xi: \Omega \Ra \R\) --- случайная величина \(\Lra \{\xi \le x\} \in \mathcal{F} \Lra \{\xi < x\} \in \mathcal{F}\)
        \item \(\xi = (\xi_1, \dots, \xi_n)\) --- случайный вектор \(\Lra \forall i: \xi_i\) измеримо
    \end{enumerate}
\end{corollary}
\begin{proof}
    \begin{enumerate}
        \item Очевидно следует из критерия, т.к. лучи (замкнутые или открытые) порождают \(\mathcal{B}(\R)\).
        \item В одну сторону. возьмем \(B_i \in \mathcal{B}(\R)\). Т.к. \(\underbrace{\R \times \dots \times \R \times}_{i - 1}B_i \times \dots \times \R= B \in \mathcal{B}(\R^n)\), то \(\{\xi_i \in B_i\} = \{\xi \in B\} \in \mathcal{F}\). В другую: \(\{\xi \in B_1 \times \dots \times B_n\} = \bigcap_{i = 1}^n\{\xi_i \in B_i\}\). Далее, по критерию измеримости, следствие верно (т.к. полуалгебра кубов порождает \(\mathcal{B}(\R^n)\)).
    \end{enumerate}
\end{proof}

\subsection{Действия со случайными величинами}
\begin{enumerate}
    \item Если \(f: \R^n \ra \R^l\) --- борелевская, \(\xi\) --- случайная, то \(f(\xi)\) --- случайная.
    \item Арифметические операции
    \item Пусть \(\xi_n\) --- последовательность случайных величин, тогда \(\sup_n \xi_n, \inf_n \xi_n, \limsup \xi_n, \liminf \xi_n\) --- тоже случайные величины
\end{enumerate}

\subsection{Характеристики случайных величин и векторов}
\begin{definition}
    Распределением случайной величины \(\xi\) на \((\Omega, \mathcal{F}, P)\) называется вероятностная мера \(P_\xi\) на \((\R, \mathcal{B}(\R))\), определяемая по правилу \(\forall B \in \mathcal{B}(\R): P_\xi(B) = P(\xi \in B)\)
\end{definition}

\begin{definition}
    Распределением случайного вектора \(\xi\) на \((\Omega, \mathcal{F}, P)\) называется вероятностная мера \(P_\xi\) на \((\R^n, \mathcal{B}(\R^n))\), определяемая по правилу \(\forall B \in \mathcal{B}(\R^n): P_\xi(B) = P(\xi \in B)\)
\end{definition}

\begin{exercise}
    \(P_\xi\) действительно является вероятностной мерой.
\end{exercise}

\begin{definition}
    Функция распределения случайной величины \(\xi\) --- это функция \(F_\xi(x) = P_\xi((-\infty, x]) = P(\xi \le x)\).
\end{definition}

\begin{definition}
    Функция распределения случайного вектора \(\vec{\xi}\) --- это функция \(F_{\vec{\xi}} = P_{\vec{\xi}}((-\infty, \vec{x}]) = P(\vec{\xi} \le \vec{x}) = P(\xi_1 \le x_1, \xi_2 \le x_2, \dots \xi_n \le x_n)\).
\end{definition}

\begin{note}
    Обозначение \(\{\xi \le x, \eta \le y\}\) обозначает \(\{\xi \le x\} \cap \{\eta \le y\}\).
\end{note}

\begin{note}
    Случайные величины и векторы наследуют классификацию распределений.
\end{note}

\begin{definition}
    \(\sigma\)-алгебра, порожденная случайной величиной \(\xi\) называется \(\mathcal{F}_\xi = \{\{\xi \in B\}: B \in \mathcal{B}(\R)\}\).
\end{definition}

\begin{definition}
    \(\sigma\)-алгебра, порожденная случайным вектором \(\xi\) называется \(\mathcal{F}_\xi = \{\{\xi \in B\}: B \in \mathcal{B}(\R^n)\}\).
\end{definition}

\begin{exercise}
    Проверить, что \(\mathcal{F}_\xi\) действительно является \(\sigma\)-алгебой.
\end{exercise}

\begin{definition}
    Пусть \(\xi\) --- случайный вектор на \((\Omega, \mathcal{F}, P)\). Пусть \(\mathcal{C} \subset \mathcal{F}\) --- некая \(\sigma\)-алгебра. Тогда \(\xi\) является \(\mathcal{C}\)-измеримой, если \(\mathcal{F}_\xi \subset \mathcal{C}\).
\end{definition}

Возникает вопрос: а что, если \(\mathcal{C} = \mathcal{F}_\eta\)?
\begin{lemma}
    Случайная величина \(\xi\) является \(\mathcal{F}_\eta\)-измеримой \(\Lra \exists\) борелевская функция \(f\), такая, что \(\xi = f(\eta)\).
\end{lemma}
\begin{proof}\indent
    \begin{enumerate}
        \item [\(\La\)] Пусть \(B \in \mathcal{B}(\R)\). Тогда \(\{\xi \in B\} = \{f(\eta) \in B\} = \{\eta \in \underbrace{f^{-1}(B)}_{\in \mathcal{B}(\R)}\} \in \mathcal{F}_\eta\).
        \item [\(\Ra\)] Пусть \(\xi = I_A, A \in \mathcal{F}_\eta\). Тогда \(\exists B \in \mathcal{B}(\R^n)\), т.ч. \(A = \{\eta \in B\}\). Тогда \(\xi = f(\eta), f(x) = I_B\). Следовательно, все простые \(\mathcal{F}_\eta\)-измеримые случайные величины тоже являются борелевскими функциями от \(\eta\). Если \(\xi\) --- произвольная случайная величина, то рассмотрим последовательность последовательность величин \(\{\xi_n\}\), такую, что \(\forall \omega \in \Omega: \xi_n(\omega) \ra \xi(\omega)\), где \(\xi_m\) --- простые случайные величины и \(\forall m \in \N: \xi_m = g_m(\xi)\), где \(g\) --- борелевская. Далее, \(\mathcal{F}_{\xi_m} \subset \mathcal{F}_{\xi} \subset \mathcal{F}_{\eta} \Ra \exists \) борелевская функция \(f_m\), т.ч. \(\xi_m = f_m(\eta)\). Введем \(B = \{x \in \R^n: \exists \lim_{m \ra \infty} f_m(x)\}\). Положим \(f(x) = \left\{\begin{array}{l}
            \lim_{m \ra \infty} f_m(x), x \in B \\
            0, x \notin B.
        \end{array}\right.\). Заметим, что \(B \in \mathcal{B}(\R^n) \Ra f\) --- тоже борелевская функция. По построению \(\forall \omega \in \Omega: \xi(\omega) = \lim_{m \ra \infty}\xi_m(\omega) = \lim_{m \ra \infty}f_m(\eta(\omega)) = f(\eta)\).
    \end{enumerate}
\end{proof}

\section{Математическое ожидание}
Пусть \(\xi\) --- случайная величина на \((\Omega, \mathcal{F}, P)\)
\begin{definition}
    Математическое ожидание случайной величины \(\xi\) называется интеграл Лебега:
    \[\E\xi = \int_{\Omega} \xi(\omega) P(d\omega) \int_{\Omega} \xi dP\]
\end{definition}

\begin{reminder}
    Напомним, как мы строили интеграл Лебега:
    \begin{enumerate}
        \item Для простой функции как сумму
        \item Для неотрицательной функции как предел
        \item Для функции разных знаков, разбиваем \(\xi = \xi^+ - \xi^-\)
    \end{enumerate}
    Далее говорим следующее:
    \begin{enumerate}
        \item Если \(\E \xi^{\pm}\) конечно, то \(\E\xi = \E\xi^+ - \E\xi^-\).
        \item Если одно из \(\E\xi^{\pm}\) равно \(+\infty\), то \(\E\xi = \pm\infty\) (ставим соответствующий знак).
        \item Если \(\E\xi^+ = \E\xi^- = +\infty\), то говорим, что \(\E\xi\) не определено.
    \end{enumerate}
\end{reminder}

\subsection{Свойства матожидания}
Из свойств интеграла:
\begin{enumerate}
    \item Линейность: \(\E(\alpha\xi + \beta\eta) = \alpha\E\xi + \beta\E\eta\)
    \item Сохраняет отношение порядка: \(\xi \le \eta \Ra \E\xi \le \E\eta\).
    \item \(|\E\xi| \le \E|\xi|\)
\end{enumerate}

\begin{definition}
    Событие \(A \in \mathcal{F}\) происходит почти наверное, если \(P(A) = 1\). Мы будем писать это как ''\(A\) п.н.''.
\end{definition}

И еще свойства математического ожидания:
\begin{enumerate}
    \setcounter{enumi}{3}
    \item Если \(\xi \ge 0, \E\xi = 0\), то \(\xi = 0\) п.н.
    \item Если \(\xi = \eta\) п.н., то \(\E\xi = \E\eta\).
    \item Если \(\forall A \in \mathcal{F}: \E(\xi \cdot I_A) \le \E(\eta \cdot I_A) \Ra \xi \le \eta\) п.н.
\end{enumerate}

\begin{theorem}[Замена переменных в интеграле Лебега]
    Пусть \(\xi\) --- случайный вектор из \(\R^n\) на \((\Omega, \mathcal{F}, P)\). Пусть \(f: \R^n \ra \R\) --- борелевская функция. Тогда \(\E f(\xi) = \int_{\R^n} f(x) P_\xi(dx)\).
\end{theorem}
\begin{proof}
    Пусть сначала \(f(x) = I_B(x), B \in \mathcal{B}(\R^n)\). Тогда
    \[\E f(\xi) = E \cdot I\{\xi \in B\} = P(\xi \in B) = P_\xi(B) = \int_B P\xi(dx) = \int_{\R^n} I_B(x) P_\xi(dx) = \int_{\R^n} f(x) P_\xi(dx)\]
    Заключаем, что теорема верна для простых функцйи в силу линейности матожидания и интеграла Лебега. Далее осуществляем приближение простыми функциями в общем случае.
\end{proof}
