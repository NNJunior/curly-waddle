
\lecture{1}
\section{Продолжение автономных систем}

\begin{reminder}
    Мы рассматриваем автономные системы, т.е. вида \(x' = f(x)\).
\end{reminder}

\subsection{Линейные системы}
Будем рассматривать уравнение в \(\R^2\):
\begin{equation}
    x' = Ax
\end{equation}
Где \(A \in \R^{2\times 2}\) --- матрица, причем \(\det A \ne 0\) (система простая)

\subsubsection{Вещественные собственные числа у матрицы}

Для начала рассмотрим случай наличия двух собственных векторов у матрицы \(A\). Положим \(h_1, h_2\) --- собственные векторы, тогда они ЛНЗ. Заметим, что тогда:
\[x(t) = c_1e^{\lambda_1t}h_1 + c_2e^{\lambda_2t}h_2\]

Рассмотрим плоскость в базисе \((h_1, h_2)\). Пусть \(\xi_i\) --- \(i\)-ая координатная функция решения. Тогда:
\[\xi_1 = c_1e^{\lambda_1t}, \xi_2 = c_2e^{\lambda_2t} = c_2e^{\frac{\lambda_2}{\lambda_1}\lambda_1t} = c_2\left( \frac{\xi_1}{c_1} \right)^{\frac{\lambda_2}
{\lambda_1}}\]

Таким образом, в координатах \(h_1, h_2\) и при \(t \ra \infty\), мы можем видеть следующую картинку (стрелки показывают движение решения при \(t \ra \infty\)):

\paragraph{Случай \(\lambda_1 < 0, \lambda_2 < 0, |\lambda_1| < |\lambda_2|\)}

В таком случае поведение траектории в окрестности положения равновесия называется \textit{устойчиывм узлом}. Картинка в данном случае будет следующая:

\begin{center}
    тут должна быть картинка
\end{center}

\paragraph{Случай \(\lambda_1 > 0, \lambda_2 > 0, \lambda_1 < \lambda_2\)}
В таком случае поведение траектории в окрестности положения равновесия называется \textit{неустойчивым узлом}. Картинка в данном случае будет следующая:

\begin{center}
    тут должна быть картинка
\end{center}

\paragraph{Случай \(\lambda_1 < 0 < \lambda_2\)}
В таком случае поведение траектории в окрестности положения равновесия называется \textit{седлом}. Картинка в данном случае будет следующая:

\begin{center}
    тут должна быть картинка
\end{center}

\paragraph{Случай \(\lambda_1 = \lambda_2\)}
В таком случае поведение траектории в окрестности положения равновесия называется \textit{дикритическим узлом}. Картинка в данном случае будет следующая:

\begin{center}
    тут должна быть картинка
\end{center}

Для случая, когда собственный вектор один:

\paragraph{Случай одного собственного вектора с собственным числом \(\lambda\)}
Тогда пусть \(h_1\) --- собственный вектор, \(h_2\) --- присоединенный к нему. Тогда:
\[x(t) = c_1e^{\lambda t}h_1 + c_2e^{\lambda t}(h_1t + h_2) = h_1\underbrace{(c_1e^{\lambda t} + c_2e^{\lambda t}t)}_{\xi_1} + h_2 \underbrace{c_2e^{\lambda t}}_{\xi_2}\]

Тогда имеем:
\[\xi_1 = \frac{c_1}{c_2}\xi_2 + \xi_2 \frac{1}{\lambda}\ln\frac{\xi_2}{c_2}\]

В таком случае поведение траектории в окрестности положения равновесия называется \textit{дикритическим узлом}. Картинка в данном случае будет следующая:

\begin{center}
    тут должна быть картинка
\end{center}

\subsubsection{Комплекснозначные собственные числа у матрицы}
Пусть \(\lambda = \alpha + i\beta, \beta \ne 0\). Тогда \(h = h_1 \pm ih_2\) --- собственные векторы, где \(h_1, h_2\) --- ЛНЗ. Тогда:
\[x(t) = ce^{\lambda t}h + \overline{c}e^{\overline{\lambda}t}\overline{h}, c \in \Cm\]
Положим \(c = \frac{r}{2}e^{i\phi}\), тогда:
\[x(t) = \frac{r}{2}e^{\alpha t}\left( e^{i\phi + i\beta t}(h_1 + ih_2) e^{-i\phi - i\beta t}(h_1 - ih_2) \right) = re^{\alpha t}(\cos(\phi + \beta t)h_1 - \sin(\phi + \beta t)h_2)\]
Получаем:
\[\xi_1(t) = re^{\alpha t}\cos(\phi + \beta t)\]
\[\xi_2(t) = re^{\alpha t}\sin(\phi + \beta t)\]

Картинки в зависимости от ранзых \(\alpha, \beta\) будут следующие:
\begin{center}
    тут должны быть картинки
\end{center}

\subsection{Нелинейные системы}
Будем рассматривать уравнение
\begin{equation}
    x' = f(x), f: \Omega \ra \R^2, f \in C^2, \Omega\text{ --- открытое}
\end{equation}
Пусть \(\tilde{x} \in \Omega\) таково, что \(f(\tilde{x}) = 0\), т.е. \(\tilde{x}\) --- положение равновесия. Рассмотрим еще одно уравнение:
\begin{equation}
    y' = f'(\tilde{x})y, f'(x) = \left( \begin{array}{cc}
        \frac{\partial f_1}{\partial x_1}(x) & \frac{\partial f_1}{\partial x_2}(x) \\ 
        \frac{\partial f_2}{\partial x_1}(x) & \frac{\partial f_2}{\partial x_2}(x) \\ 
    \end{array} \right)
\end{equation}

\begin{theorem}[б/д]
    Пусть \(\det f(\tilde{x}) \ne 0\), (т.е. система (1.3) простая), причем \(0\) не является ее центром. Тогда \(\exists U\) --- окрестность \(\tilde{x}\), \(\exists V\) --- окрестность \(0\), \(\exists \psi: U \ra V\) --- гомеоморфизм, такие, что выполнены следующие условия:
    \begin{enumerate}
        \item \(\forall\) траектории \(X \subset V\) системы (1.2), \(\psi(X)\) --- тракетория системы (1.3)
        \item \(\forall\) траектории \(Y \subset U\) системы (1.3), \(\psi^{-1}(Y)\) --- тракетория системы (1.2)
    \end{enumerate}
\end{theorem}

\begin{example}
    Рассмотрим систему:
    \begin{equation*}
        \begin{cases*}
            x_1' = -x_2 - x_1|x| \\
            x_2' = x_1 - x_2|x|
        \end{cases*}, \tilde{x} = \left( \begin{array}{c}
            0 \\
            0
        \end{array} \right)
    \end{equation*}

    Тогда:
    \[f'(0) = \left( \begin{array}{cc}
        0 & -1 \\
        1 & 0
    \end{array} \right)\]
    И линеаризованная система имеет вид:
    \[y' = \left( \begin{array}{cc}
        0 & -1 \\
        1 & 0
    \end{array} \right)y\]
    Заметим, что тогда \(\lambda \pm i\) и \(0\) --- центр. Сделаем замену \(x_1 = r\cos\phi, x_1 = r\sin\phi\), имеем:
    \begin{eqnarray*}
        \begin{cases*}
            r'\cos\phi - r(\sin\phi)\phi' = -r\sin\phi - r^2\cos\phi \\
            r'\sin\phi + r(\cos\phi)\phi' = r\cos\phi - r^2\sin\phi
        \end{cases*}
    \end{eqnarray*}
    \begin{eqnarray*}
        \begin{cases*}
            r' = -r^2
            -r\phi' = -r \Ra \phi' = 1
        \end{cases*}
    \end{eqnarray*}
    Получили, что \(\phi = \phi_0 + t\) и 
    \[x(t) = \left( \begin{array}{c}
        r(t)\cos(\phi_0 + t) \\
        r(t)\sin(\phi_0 + t) \\
    \end{array} \right)\]
    Таким образом, картинка будет следующей:
    \begin{center}
        тут должна быть картинка
    \end{center}
    Т.е. одна траектория получилась не замкнутой. Но тогда между ними не может существовать гомеоморфизма. Таким образом, условие, что 0 --- не центр существенно.
\end{example}

\lecture{2}

\section{Устойчивость решений дифференциальных уравнений}
Пусть \(\Omega \subset \R^n\) открыто, \(f: \Omega \to \R^n, f \in C^1\). Будем рассматривать уравнение:

\begin{equation}
    x' = f(x)
\end{equation}

Пусть \(\phi(\cdot, \xi)\) --- непрерывное решение задачи Коши
\begin{equation*}
    \begin{cases}
        \dot{x} = f(x) \\
        x(0) = \xi
    \end{cases}
\end{equation*}

\begin{definition}
    Решение \(\phi(\cdot, \hat{\xi})\) называется устойчивым по Ляпунову, если 
    \begin{enumerate}
        \item \(\exists r > 0: \forall \xi \in O(\hat{\xi}, r) \ \phi(\cdot, \xi) \) определена на \([0, +\infty)\)
        \item \(\forall \epsilon > 0 \exists \delta > 0: \xi \in O(\hat{\xi}, \delta) \Ra \phi(t, \xi) \in O(\phi(t, \hat{\xi}), \epsilon) \forall t \in [0, +\infty)\)
    \end{enumerate}
\end{definition}

\begin{definition}
    Решение \(\phi(\cdot, \hat{\xi})\) называется асимптотически устойчивым, если оно устойчиво по Ляпунову и \(\exists d > 0: \forall \xi \in O(\hat{\xi}, d): |\phi(t, \xi) - \phi(t, \hat{\xi})| \ra 0\) при \(t \ra 0\)
\end{definition}

\subsection{Примеры}

\begin{enumerate}
    \item \textbf{Логистическое уравнение.} 
    \[x' = rx\left( 1 - \frac{x}{k} \right), \phi(t, \xi) = \frac{k\xi e^{rt}}{k - \xi + \xi e^{rt}}\]

    \begin{center}
        тут должна быть картинка
    \end{center}

    Тогда: при \(\hat{\xi} = k, \phi(\cdot, \hat{\xi})\) --- асимптотически устойчиво, а при \(\hat{\xi} = 0\) оно не устойчиво по Ляпунову.

    \item 
    \[x' = Ax, A \in \R^{2 \times 2}\]
    Тогда в зависимости от картинки решение будет иметь разный характер устойчивости:
    \begin{itemize}
        \item Седло \(\Ra\) не устойчиво по Ляпунову
        \item Центр \(\Ra\) устойчиво по Ляпунову, но не асимптотически
        \item Устойчивый узел \(\Ra\) устойчиво асимптотически
    \end{itemize}
\end{enumerate}

\begin{theorem}
    Пусть \(A \in \R^{n \times n}, \lambda_1 \dots \lambda_m\) --- собственные числа матрицы \(A\). Тогда:
    \begin{enumerate}
        \item Если \(\Re \lambda_j < 0 \forall j\) то \(0\) асимптотически устойчив
        \item Если
        \begin{itemize}
            \item \(\Re \lambda_j \le 0 \forall j \Ra 0\)
            \item \(\exists j: \Re \lambda_j = 0\)
            \item \(\Re \lambda_j = 0 \Ra\) все соответствующие жордановы клетки имеют размер 1.
        \end{itemize}
        то \(0\) устойчив по Ляпунову и не устойчив асимптотически.
        \item Во всех остальных случаях \(0\) не устойчив по Ляпунову.
    \end{enumerate}
\end{theorem}
\begin{proof}
    Нам известно, что:
    \begin{equation}
        x(t) = P_1(t)e^{\lambda_1t} + \dots + P_m(t)e^{\lambda_mt}
    \end{equation}
    Причем \(\deg P_j <\) размер наибольшей Жордановой клетки, соответствующей числу \(\lambda_j\). Пусть \(\lambda_j = \alpha_j + i\beta_j\)
    \begin{enumerate}
        \item Пусть \(X(t)\) --- ФМР, \(X(0) = I\). Пусть \(x(t)\) --- один из столбцов ФМР. Тогда:
        \[x(t) = \sum_{j = 1}^m P_j(t)e^{\lambda_jt}(\cos(\beta_jt) + i\sin(\beta_j t)) \ra 0, t \ra +\infty\]
        Тогда
        \begin{equation}
            X(t) \ra 0, t \ra +\infty \Ra \|X(t)\| \ra 0, t \ra +\infty
        \end{equation}
        Получаем, что \(\exists \gamma > 0: \|X(t)\| \le \gamma \forall t \ge 0\).
        \[\phi(t, \xi) = X(t)\xi, t \ge 0, \xi \in \R^n\]
        \(\forall \epsilon > 0 \) положим \(\delta = \frac{2\epsilon}{\gamma}\). Тогда при \(\xi \in O(0, \delta)\), то \(|\phi(t, \xi)| = |X(t)\xi| \le \|X(t)\| \cdot |\xi| \le \gamma \delta < \epsilon\). Таким образом, данное решение асимптотически устойчиво.

        \item Из размера жордановых клеток и условия, что \(\Re lambda_j \le 0\) получаем, что каждый элемент матрицы \(X(t)\) ограничен \(\Ra \exists \gamma > 0: \|X(t)\| \le \gamma\). Тогда аналогично получаем, что \(0\) устойчиво по Ляпунову.
        
        Докажем, что нет асимптотической устойчивости. Пусть \(\alpha_1 = \Re \lambda_1 = 0, h_1 \in C^n\) --- собственный вектор. Тогда \(x(t) = e^{\lambda_1t}h_1 = (\cos(\beta_1 t) + i\sin(\beta_1 t))h_1\) --- решение. Тогда решением также являются функции: \(x_k(t) = \frac{1}{k} \Re x(t), k \in \N\). Тогда \(\forall d > 0 \exists k: x_k(0) = \frac{1}{k}\Re(x(0)) \in O(0, d)\). Но \(x_k(t) \not\ra 0\), т.к.:
        \[x_k(t) = \frac{1}{k}(\Re h_1 \cos(\beta_1 t) + \Im h_1 - \sin(\beta_1 t))\]
        
        \item Нетрудно проверить, что тогда \(\exists j: P_j(t)e^{\lambda_j t}\) не ограничено. Тогда: \(x(t) = P_j(t)e^{\lambda_j t}\) --- решение. Но тогда \(x_k(t) = \frac{1}{k} \Re x(t), y_k(t) = \frac{1}{k} \Im x(t)\) --- тоже решения. Но в таком случае, либо \(x_k\), либо \(y_k\) не ограничены. Б.О.О., \(x_k\) не ограничено. Положим \(\epsilon = 1\). \(\forall \delta > 0 \exists k: |x_k(0)| < \delta\), но \(|x_k(t)|\) не ограничена при \(t \in [0, +\infty) \Ra 0\) не является устойчивым по Ляпунову.
    \end{enumerate}
\end{proof}

\lecture{3}

\subsection{Теоремы Ляпунова об устойчивости}

Пусть \(\Omega \subset \R^n\) --- открытое, \(f: \Omega \to \R^n\), \(f \in C^1\), \(\widehat{x} \in \Omega\), \(f(\widehat{x}) = 0\).

\begin{equation}
    x' = f(x)
\end{equation}

\begin{equation}
    \begin{cases}
        x' = f(x) \\
        x(0) = \xi
    \end{cases}
\end{equation}

Пусть \(\phi(\cdot, \xi)\) --- непрерывное решение задачи Коши (2.5), \(\widetilde{\Omega} \subset \Omega\) --- открытое, \(v: \widetilde{\Omega} \to \R\), \(v \in C^1\).

\begin{definition}
    \(\displaystyle \frac{dv}{dt}(x) = \langle v'(x), f(x) \rangle\), \(x \in \widetilde{\Omega}\) --- производная в силу системы (2.4).

    Если \(x(\cdot)\) --- решение (2.4) и \(x(t) \in \widetilde{\Omega} \; \forall t\), то
    \[
    \frac{d}{dt}\bigl(v(x(t))\bigr) \equiv \langle v'(x(t)), \dot{x}(t) \rangle \equiv \langle v'(x(t)), f(x(t)) \rangle \equiv \frac{dv}{dt}(x(t)).
    \]
\end{definition}

\begin{theorem}[Ляпунова об устойчивости]
    Пусть \(\exists\) окрестность \(\widetilde{\Omega} \subset \Omega\) точки \(\widehat{x}\), \(\exists v \in C^1(\widetilde{\Omega}, \R)\) такие, что:
    \begin{enumerate}
        \item \(v(\widehat{x}) = 0\), \(v(x) > 0 \; \forall x \ne \widehat{x}\);
        \item \(\displaystyle \frac{dv}{dt}(x) \le 0 \; \forall x \in \widetilde{\Omega}\).
    \end{enumerate}
    Тогда \(\widehat{x}\) устойчиво по Ляпунову.
\end{theorem}

\begin{proof}
    Зафиксируем \(\epsilon > 0\). Б.О.О. \(B(\widehat{x}, \epsilon) \subset \widetilde{\Omega}\). Положим \(m = \min_{x \in S(\widehat{x}, \epsilon)} v(x) > 0\). Так как \(v(\widehat{x}) = 0\) и \(v\) непрерывна, то \(\exists \delta > 0\): \(v(x) < m \; \forall x \in B(\widehat{x}, \delta)\). Для любого \(\xi \in B(\widehat{x}, \delta)\) покажем, что \(\phi(t, \xi) \in B(\widehat{x}, \epsilon)\) для всех \(t \ge 0\).

    Предположим противное: пусть \(\exists T > 0\) такое, что \(\phi(T, \xi) \notin B(\widehat{x}, \epsilon)\). В силу непрерывности решения и того, что \(\phi(0, \xi) = \xi \in B(\widehat{x}, \delta) \subset B(\widehat{x}, \epsilon)\), найдётся первый момент времени \(t_0 \in (0, T]\), когда траектория выходит на границу шара, т.е.
    \[
    |\phi(t_0, \xi) - \widehat{x}| = \epsilon, \quad \text{и} \quad |\phi(t, \xi) - \widehat{x}| < \epsilon \; \forall t \in [0, t_0).
    \]
    Тогда \(\phi(t_0, \xi) \in S(\widehat{x}, \epsilon)\). По определению \(m\) имеем \(v(\phi(t_0, \xi)) \ge m\).

    С другой стороны, из условия 2 теоремы \(\bigl( \frac{dv}{dt} \le 0 \bigr)\) следует, что функция \(v(\phi(t, \xi))\) не возрастает на \([0, t_0]\). Действительно,
    \[
    \frac{d}{dt} v(\phi(t, \xi)) = \frac{dv}{dt}(\phi(t, \xi)) \le 0.
    \]
    Поэтому
    \[
    v(\phi(t_0, \xi)) \le v(\phi(0, \xi)) = v(\xi) < m,
    \]
    так как \(\xi \in B(\widehat{x}, \delta)\). Получили противоречие: \(v(\phi(t_0, \xi)) \ge m\) и \(v(\phi(t_0, \xi)) < m\) одновременно.

    Следовательно, наше предположение неверно, и \(\phi(t, \xi) \in B(\widehat{x}, \epsilon)\) для всех \(t \ge 0\). В частности, отсюда вытекает, что решение \(\phi(\cdot, \xi)\) определено на \([0, +\infty)\) (оно не может покинуть компакт \(B(\widehat{x}, \epsilon) \subset \widetilde{\Omega}\), где выполнены условия теоремы существования и единственности). Таким образом, мы нашли \(\delta > 0\) (то самое, из непрерывности \(v\)) такое, что для любого \(\xi \in B(\widehat{x}, \delta)\) выполнено \(\phi(t, \xi) \in B(\widehat{x}, \epsilon) \subset O(\phi(t, \widehat{x}), \epsilon)\) для всех \(t \ge 0\) (напомним, что \(\phi(t, \widehat{x}) = \widehat{x}\) в силу \(f(\widehat{x}) = 0\)).

    Это в точности означает устойчивость по Ляпунову положения равновесия \(\widehat{x}\).
\end{proof}

\begin{example}
    Рассмотрим одномерное автономное уравнение
    \[
    \dot{x} = f(x), \quad f \in C^1(\Omega), \quad \tilde{x} \in \Omega, \quad f(\tilde{x}) = 0.
    \]
    Пусть существует окрестность \(U \subset \Omega\) точки \(\tilde{x}\) такая, что для всех \(x \in U \setminus \{\tilde{x}\}\) выполнено
    \[
    (x - \tilde{x}) f(x) < 0.
    \]
    Определим функцию \(v(x) = (x - \tilde{x})^2\). Очевидно,
    \[
    v(\tilde{x}) = 0, \quad v(x) > 0 \quad \forall x \neq \tilde{x}.
    \]
    Её производная в силу системы равна
    \[
    \frac{dv}{dt}(x) = 2(x - \tilde{x}) f(x) < 0 \quad \forall x \in U \setminus \{\tilde{x}\}.
    \]
    Таким образом, \(v\) удовлетворяет условиям теоремы Ляпунова об асимптотической устойчивости, и \(\tilde{x}\) является асимптотически устойчивым положением равновесия.
\end{example}

\begin{theorem}[Ляпунова об асимптотической устойчивости]
    Пусть \(\exists\) окрестность \(\widetilde{\Omega} \subset \Omega\) точки \(\tilde{x}\), \(\exists v \in C^1(\widetilde{\Omega}, \R)\) такие, что:
    \begin{enumerate}
        \item \(v(\tilde{x}) = 0\), \(v(x) > 0 \; \forall x \ne \tilde{x}\);
        \item \(\displaystyle \frac{dv}{dt}(x) < 0 \; \forall x \in \widetilde{\Omega} \setminus \{\tilde{x}\}\).
    \end{enumerate}
    Тогда \(\tilde{x}\) асимптотически устойчиво.
\end{theorem}

\begin{proof}
    По теореме Ляпунова об устойчивости, \(\tilde{x}\) устойчиво \(\Rightarrow \exists \epsilon > 0, \exists \delta > 0: B(\tilde{x}, \epsilon) \subset \widetilde{\Omega}\) и \(\forall \xi \in B(\tilde{x}, \delta)\) \(\forall t \ge 0\) \(\phi(t, \xi) \in B(\tilde{x}, \epsilon)\). 
    
    Рассмотрим \(\xi \in B(\tilde{x}, \delta)\) и докажем, что \(\phi(t, \xi) \to \tilde{x}\) при \(t \to +\infty\). Предположим противное: пусть \(\exists \eta > 0\) и последовательность \(\{t_j\} \to +\infty\) такие, что \(|\phi(t_j, \xi) - \tilde{x}| \ge \eta\) для всех \(j\). Б.О.О. можно считать \(\eta < \epsilon\). 
    
    Множество \(K = \{x \in \overline{B(\tilde{x}, \epsilon)} : \eta \le |x - \tilde{x}| \le \epsilon\}\) компактно и не содержит \(\tilde{x}\). Функция \(v\) непрерывна и положительна на \(K\), поэтому достигает минимума:
    \[
    \mu = \min_{x \in K} v(x) > 0.
    \]
    В частности, \(v(\phi(t_j, \xi)) \ge \mu\) для всех \(j\).
    
    Так как \(\frac{dv}{dt}(x) < 0\) для всех \(x \in \widetilde{\Omega} \setminus \{\tilde{x}\}\), функция \(v(\phi(t, \xi))\) строго убывает по \(t\). Поскольку она ограничена снизу нулём, существует предел
    \[
    l = \lim_{t \to +\infty} v(\phi(t, \xi)) \ge \mu > 0.
    \]
    
    Рассмотрим множество \(\Omega_l = \{x \in \overline{B(\tilde{x}, \epsilon)} : v(x) \ge l\}\). Оно замкнуто, не содержит \(\tilde{x}\) (ибо \(v(\tilde{x}) = 0\)), и компактно. На этом множестве \(\frac{dv}{dt}\) непрерывна и отрицательна, следовательно, достигает максимума (т.е. существует отрицательная верхняя грань):
    \[
    -\alpha = \max_{x \in \Omega_l} \frac{dv}{dt}(x) < 0, \quad \alpha > 0.
    \]
    Но тогда для всех \(t\), поскольку \(\phi(t, \xi) \in \Omega_l\) (ибо \(v(\phi(t, \xi)) \ge l\)), имеем
    \[
    \frac{d}{dt} v(\phi(t, \xi)) = \frac{dv}{dt}(\phi(t, \xi)) \le -\alpha.
    \]
    Интегрируя от \(0\) до \(t\), получаем
    \[
    v(\phi(t, \xi)) \le v(\xi) - \alpha t.
    \]
    При достаточно больших \(t\) правая часть становится отрицательной, что противоречит неотрицательности \(v\). 
    
    Полученное противоречие означает, что наше предположение неверно, и \(\phi(t, \xi) \to \tilde{x}\) при \(t \to +\infty\). Таким образом, \(\tilde{x}\) асимптотически устойчиво.
\end{proof}

\begin{example}
    Рассмотрим систему:
    \[
    \begin{cases}
        x_1' = x_2 - x_1, \\
        x_2' = -x_1^3.
    \end{cases}
    \]
    Единственное положение равновесия — \(\tilde{x} = (0, 0)\). Определим функцию
    \[
    v(x) = (x_2 - x_1)^2 + x_2^2.
    \]
    Очевидно, \(v(0,0)=0\) и \(v(x)>0\) при \(x \neq 0\). Вычислим производную в силу системы:
    \[
    \frac{dv}{dt}(x) = 2(x_1-x_2)(x_2-x_1) + 2(2x_2-x_1)(-x_1^3) = -2(x_1-x_2)^2 - 4x_1^3x_2 + 2x_1^4.
    \]
    Квадратичная часть \(-2(x_1-x_2)^2\) неположительна и обращается в нуль лишь при \(x_1=x_2\). На этой прямой (подставим \(x_2=x_1\)):
    \[
    \frac{dv}{dt}(x_1,x_1) = -2x_1^4 < 0 \quad (x_1 \neq 0).
    \]
    Таким образом, в достаточно малой окрестности нуля \(\frac{dv}{dt}(x) < 0\) для всех \(x \neq 0\). По теореме Ляпунова об асимптотической устойчивости \(\tilde{x}\) асимптотически устойчиво.
\end{example}