
\hypertarget{lecture1}{}


\section{Пространство Лебега}
\subsection{Пространства \(L_p\)}
\begin{definition}
    Пусть \(E \subset \R^m\). Будем говорить, что \(f: E \ra \Cm\) измерима (интегрируема) по Лебегу, если \(\Re f, \Im f\) измеримы (интегрируемы) по Лебегу.
\end{definition}

\begin{definition}
    В случае интегрируемости положим \(\int_E f = \int_E \Re f + i \int_E \Im f\). Полученный интеграл линеен и аддитивен по множествам
\end{definition}

\begin{note}
    \[\left\|\int_E f \right\| \le \int_E |f|\]
\end{note}
\begin{proof}
    \[\int_E f = \left\|\int_E f\right\| e^{i \theta} \Ra \int_E fe^{-i \theta} = \left\|\int_E f\right\| = \int_E \Re(e^{-i\theta} f) \le \int_E |e^{i\theta} f| = \int_E |f|\]
\end{proof}

\begin{definition}
    Пусть \(1 \le p \le < \infty\) и \(E\) измеримо. Определим:
    \[L_p(E) = \{f: E \ra \Cm | \int_E |f|^p < \infty\}\]
    В таком случае положим \(\|f\|_p = \left( \int_E |f|^p \right)^{\frac{1}{p}}\)
\end{definition}

Пусть \(f, g \in L_p(E)\). Если \(\lambda \in \Cm\), то \(\lambda f \in L_p(E)\) и ввиду \(|f + g|^p \le (2\max\{|f|, |g|\})^p \le 2^p(|f|^p + |g|^p)\) выполнено \(f + g \in L_p\). Получили, что \(L_p\) является линейным пространством относительно \(+, \lambda\cdot\).



\begin{lemma}
    Пусть \(a, b \ge 0\). Если \(1 < p < \infty, \frac{1}{p} + \frac{1}{q} = 1\), то \(ab \le \frac{a^p}{p} + \frac{b^q}{q}\), причем равенство достигается тогда и только тогда, когда \(a^p = b^q\)
\end{lemma}
\begin{proof}
    Можно считать, что \(ab > 0\). Ввиду выпуклости экспоненты, имеем:
    \[e^{\frac{x}{p} + \frac{y}{q}} \le \frac{1}{p}e^x + \frac{1}{q}e^y\]
    Положим \(x = p\ln a, y = q\ln b\) и получаем желаемое.
\end{proof}

\begin{theorem}[Гельдер]
    Пусть \(1 < p < \infty, \frac{1}{p} + \frac{1}{q} = 1\). Если \(f \in L_p(E), g \in L_q(E)\), то \(fg \in L_1(E)\) и \(\|fg\|_1 \le \|f\|_p\|g\|_q\)
\end{theorem}
\begin{proof}
    Если \(\|f\|_p = 0 \Ra f = 0\) п.в. \(\Ra fg = 0\) п.в. \(\Ra \) утверждение доказано. Аналогично и для случая \(\|g\|_q = 0\). В противном случае, получаем:
    \[\|f\|_p\|g\|_q > 0\]
    По предыдущей лемме, имеем:
    \[\frac{|f(x)|}{\|f\|_p}\frac{|g(x)|}{\|g\|_q} \le \frac{1}{p}\left( \frac{|f(x)|}{\|f\|_p} \right)^p + \frac{1}{q}\left( \frac{|g(x)|}{\|g\|_q} \right)^q\]
    Проинтегрировав, получим:
    \[\frac{1}{\|f\|_p\|g\|_q}\int_E |fg| \le \frac{1}{p}\left( \int_E \frac{|f|^p}{\|f\|_p^p} \right) + \frac{1}{q}\left( \int_E \frac{|g|^q}{\|g\|_q^q} \right) = 1\]
\end{proof}

\begin{note}
    В неравенстве гельдера равенство имеет место тогда и только тогда, когда \(|f|^p = c|g|^q\) п.в. на \(E\) для некоторого \(c > 0\).
\end{note}

\begin{theorem}[Минковский]
    Пусть \(1 \le p < \infty\). Если \(f, g \in L_p(E)\), то \(f + g \in L_p(E)\) и \(\|f + g\|_p \le \|f\|_p + \|g\|_p\).
\end{theorem}
\begin{proof}
    При \(p = 1\) применим \(|f + g| \le |f| + |g|\). Далее при \(p \ge 1\): Применим неравенство Гельдера для \(p, q = \frac{p}{p - 1} \Ra \frac{1}{p} + \frac{1}{q} = 1\). 
    \[\|f + g\|_p^p = \int_E |f + g|^p \le \int_E |f||f + g|^{p - 1} + \int_E |g||f + g|^{p - 1} \le\]
    \[\le \left( \int_E |f|^p \right)^{\frac{1}{p}}\left( \int_E |f + g|^p \right)^{\frac{1}{q}} + \left( \int_E |g|^p \right)^{\frac{1}{p}}\left( \int_E |f + g|^p \right)^{\frac{1}{q}} = \]
    \[ = \|f + g\|_p^{p - 1}(\|f\|_p + \|g\|_p)\]
\end{proof}

\begin{note}
    Равенство в теореме Минковского выполняется тогда и только тогда, когда \(f = cg\) п.в. для некоторой \(c > 0\)
\end{note}

\begin{definition}
    На \(L_p\) введем отношение \(\sim\). Будем говорить, что \(f \sim g \Lra f = g\) п.в. на \(E\).
\end{definition}

\begin{note}
    \(\sim\) --- отношение эквивалентности на \(L_p\), согласованное с операциями сложения и умножения на скаляр. Факторпространство \(L_p(E)/_\sim\) будем также обозначать \(L_p(E)\)
\end{note}

\begin{corollary}
    Пространство \(L_p\) относительно нормы \(\|\cdot\|_p\) является нормированным линейным пространством
\end{corollary}
\begin{proof}\indent
    \begin{enumerate}
        \item \(\|f\|_p \ge 0, \|f\|_p = 0 \Lra f \sim 0\)
        \item \(\|\lambda f\|_p = \lambda \|f\|_p\)
        \item \(\|f + g\|_p \le \|f\|_p + \|g\|_p\)
    \end{enumerate}
\end{proof}

\begin{problem}
    Если \(\mu(E) < \infty, 1 \le p < q < \infty\), то \(L_q(E) \subsetneq L_p(E)\)
\end{problem}

\begin{reminder}
    Полное метрическое пространство --- такое, что любая фундаментальная последовательность сходится
\end{reminder}

\begin{theorem}[Рисса]
    Пространство \(L_p\) банахово (т.е. является полным относительно метрики, порожденной \(p\)-нормой). 
\end{theorem}
\begin{proof}
    Будет позднее
\end{proof}

\begin{reminder}
    Напомним, что \(\supp(f) = \overline{\{x: f(x) \ne 0\}}\). Будем называть функции с компактным носителем финитными.
\end{reminder}

\begin{lemma}
    Пусть \(f \in L_p, \epsilon > 0\). Тогда \(\exists\) простая финитная функция \(\phi\) такая, что \(\|f - \phi\|_p < \epsilon\).
\end{lemma}
\begin{proof}
    Можно считать, что \(f\) вещественнозначная (иначе приближаем \(\Re f, \Im f\)) отдельно. Т.к. \(|f - I_{B_k(0)}|^p \le |f|^p\). Тогда по Теореме Лебега о мажорируемой сходимости, \(\|f - fI_{B_k(0)}\|_p \ra 0, k \ra \infty\). Заменяя функцию \(f\) на \(fI_{B_k(0)}\) для достаточно большого \(k\), заключаем, что она финитная. Пусть сначала \(f \ge 0\). По теореме о приближении, найдется последовательность \(\{\phi_k\}\) --- простых функций, т.ч. \(0 \le \phi_1 \le \dots \), \(\phi_k \ra f\). Т.к. \(|f - \phi_k|^p \le |f|^p\). По теореме Лебега о мажорируемой сходимости \(f - \phi_k \ra 0, k \ra \infty\), причем все \(\phi_k\) финитны, т.к. \(0 \le \phi_k \le f\) на \(E\). Пусть \(f\) теперь произвольного знака \(\Ra f = f^+ - f^-\). Тогда по доказанному найдутся простые финитные функции \(\phi^+, \phi^-\), такие, что \(\|f^+ - \phi^+\|_p < \frac{\epsilon}{2}, \|f^- - \phi^-\|_p < \frac{\epsilon}{2}\). Положим \(\phi = \phi^+ - \phi^-\). Тогда \(\phi\) --- простая финитная функция и по неравенству треугольника:
    \[\|f - \phi\|_p \le \|f^+ - \phi^+\| + \|f^- - \phi^-\| < \epsilon\]
\end{proof}

\begin{theorem}
    Пусть \(f \in L_p, \epsilon > 0\). Тогда \(\exists g \in C_c^\infty(\R^m)\), т.ч. \(\|f - g\|_p < \epsilon\)
\end{theorem}
\begin{proof}
    По предыдущей лемме, любую функцию можно приблизить финитной простой функцией. Всякая простая функция есть линейная комбинация индикаторов. Из этого заключаем, что достаточно доказать теорему для случая \(f = I_A\), где \(A\) --- ограниченное измеримое множество. По свойству регулярности меры Лебега, \(\exists G, H\) --- открытые, такие, что \(G \supset A, H \supset A^c\) и \(\mu(G \setminus A) < \frac{\epsilon}{2}, \mu(H \setminus A^c) < \frac{\epsilon}{2}\). Положим \(k = H^c\) --- замкнутое и ограниченное (т.к. лежит в \(A\)), т.е. компакт, лежащий в \(A\). \(\mu(G \setminus A) \le \mu(G \setminus A) + \mu(H \setminus A^c) < \epsilon\). По теореме о гладком разбиении единицы, \(\exists g \in C^\infty(\R^m)\) с носителем в \(G\), такая, что \(0 \le g \le 1\) и \(g|_K = 1\). Поэтому \(\|I_A - g|_E\|_p^p = \int_E |I_A - g|^p \le \int_{(G \cap E) \setminus K} |I_A - g|^p \le \mu(G \setminus K) < \epsilon\)
\end{proof}

\hypertarget{lecture2}{}

\begin{theorem}[Кантора]
    Пусть \(f: U \ra \Cm\), где \(U \subset \R^n\) --- открыто. Тогда, если \(K\) --- компакт и \(f\) непрерывна на нем, то \(f\) равномерно непрерывна на нем, т.е.
    \[\forall \epsilon > 0 \exists \delta > 0 \forall x \in K \forall h \in \R^n (h < \delta \Ra |f(x) - f(x - h)| < \delta)\]
\end{theorem}
\begin{proof}
    От противного. Тогда \(\exists \{x_k\} \subset K \exists \{h_k\} \subset \R^m |f(x_k) - f(x_k - h_k)| \ge \epsilon_0, h_k \ra 0\). Т.к. \(K\) --- компакт, то \(\exists x_{n_k} \ra x_0 \in K\) и \(x_{n_k} + h_{n_k} \ra x_0\). Тогда по непрерывности, \(f(x_{n_k}) \ra f(x_0), f(x_{n_k} - h_{n_k}) \ra f(x_0) \Ra |f(x_{n_k}) - f(x_{n_k} - h_{n_k})| \ra 0\)
\end{proof}

\begin{definition}
    Пусть \(f: \R^m \ra \Cm, h \in \R^m\). Функция \(f_h: \R^m \ra \Cm, f_h(x) = f(x - h)\) называется сдвигом функции \(f\) на \(h\).
\end{definition}

\begin{note}
    Функции \(f, f_h\) одновременно лежат в \(L_p(\R^m)\) и \(\|f\|_p = \|f_h\|_p\)
\end{note}

\begin{theorem}[Непрерывность сдвига]
    Если \(f \in L_p(\R^m)\), то \(\|f_h - f\|_p \ra 0\) при \(h \ra 0\).
\end{theorem}
\begin{proof}
    Зафиксируем \(\epsilon > 0\). \(\exists g \in C_c(\R^m)\) (можно считать, что \(g = 0\) вне \(B_r(0)\)), такая, что \(\|f - g\|_p < \frac{\epsilon}{3}\). По неравенству треугольника:
    \[\|f_h - f\|_p \le \|f_h - g_h\|_p + \|g_h - g\|_p + \|g - f\|_p < \frac{2}{3}\epsilon + \|g_h - g\|\]
    Т.к. \(g\) непрерывна на \(\overline{B_r(0)}\), то \(\exists \delta \in (0, 1] \forall h \in \R^m (|h| < \delta \Ra |g(x) - g(x - h)| < \frac{\epsilon}{3M})\), где \(M^p = \mu()B_{r + 1}(0)\). Поэтому:
    \[\|g_h - g\|^p_p = \int |g(x - h) - g(x)|^pdx \le \left( \frac{\epsilon}{3M} \right)^p\mu(B_{r + 1}) = \left( \frac{\epsilon}{3} \right)^p\]
    Следовательно, при \(|h| < \delta\), выполнено \(\|f_h - f\|_p < \epsilon\)
\end{proof}

\begin{corollary}[Непрерывность сдвига для периодических функций]
    Пусть \(f\) --- \(2\pi\)-периодичная измеримая на \(\R\) функция. Если \(\int_{[-\pi, \pi]} |f(x)|^p dx < \infty\), то 
    \[\int_{[-\pi, \pi]} |f(x - h) - f(x)|^pdx \ra 0, h \ra 0\]
\end{corollary}
\begin{proof}
    Пусть \(g = \left\{\begin{array}{l}
        f, x \in (-2\pi, 2\pi) \\
        0, \text{иначе}
    \end{array}\right.\). Тогда \(g \in L_p(\R)\). Также, при \(x \in (-\pi, \pi)\) и \(|h| < \pi\) имеем: \(|g(x - h) - g(x)| = |f(x - h) - f(x)|\), а значит,
    \[\int_{[-\pi, \pi]} |f(x - h) - f(x)|^pdx = \int_{[-\pi, \pi]} |g(x - h) - g(x)|^pdx \le \|g_h - g\|_p \ra 0, h \ra 0\]
\end{proof}

\begin{theorem}[Римана об осцилляции]
    Пусть \(I\) --- промежуток в \(\R\), \(f \in L_1(I)\). Тогда 
    \[\int_I f(x)e^{i\lambda x}dx \ra 0, \lambda \ra \pm\infty\]
    В частности, 
    \[\int_I f(x)\cos xdx \ra 0, \int_I f(x)\cos xdx \ra 0, \lambda \ra \pm\infty\]
\end{theorem}
\begin{proof}
    Рассмотрим случай \(I = \R\). Сделаем в интеграле замену \(x = t - \frac{\pi}{\lambda}\), тогда:
    \[\int_I f(x)e^{i\lambda x}dx = \int_I f\left( t - \frac{\pi}{\lambda} \right)e^{i\lambda \left( t - \frac{\pi}{\lambda} \right)}dx = -\int_I f\left( t - \frac{\pi}{\lambda} \right)e^{i\lambda t}dx\]
    Следовательно,
    \[\int_I f(x)e^{i\lambda x}dx = \frac{1}{2}\left(  \int_I f(x)e^{i\lambda x}dx -\int_I f\left( x - \frac{\pi}{\lambda} \right)e^{i\lambda x}dx\right)= \frac{1}{2}\left(  \int_I \left( f(x) - f\left( x - \frac{\pi}{\lambda} \right) \right)e^{i\lambda x}dx\right)\]
    Следовательно, 
    \[\left|\int_I f(x)e^{i\lambda x}dx\right| \le \frac{1}{2}\left(  \int_I \left| f(x) - f\left( x - \frac{\pi}{\lambda} \right) \right|dx\right) = \|f_h - f\|_1 \ra 0\]
\end{proof}

\subsection{Свертка и аппроксимация функции}
\begin{definition}
    Пусть \(f, g\) --- измеримы в \(\R^m\), функция \(f * g\) определяемая по формуле
    \[(f * g)(x) = \int_{\R^m} f(x - t)g(t)dt\]
    называется сверткой функций \(f, g\).
\end{definition}

\begin{note}
    Покажем измеримость функции \(f(x - t)g(t)\) в \(\R^{2m}\). Т.к. \(g(t)\) измерима в \(\R^{2m}\), достаточно показать измеримость \(f(x - t)\). Положим \(E_a = \{x \in \R^m: f(x) < a\}\) и определим оператор \(L: \R^{2m} \ra \R^{2m}, L(x, t) = (x - t, t)\). Тогда \(\{(x - t): f(x - t) < a\} = \{(x, t): x - t \in E_a\} = L^{-1}(E_a \times \R^m)\) --- измеримо как образ измеримого при диффеоморфизме \(L^{-1}\)
\end{note}

\begin{lemma}
    Пусть \(f, g \in L_1(\R^m)\), тогда \(f * g\) определена почти всюду. Более того, \(f * g \in L_1(\R^m)\) и \(\|f * g\|_1 \le \|f\|_1\|g\|_1\)
\end{lemma}
\begin{proof}
    Определим \(H(x) = \int_{\R^m}|f(x - t)||g(t)|dt\). По теореме Тонелли:
    \[\int_{\R^m}H(x)dx = \int_{\R^m}\left( \int_{\R^m}|f(x - t)| dx\right)|g(t)|dt = \int_{\R^m}\left( \int_{\R^m}|f(y)| dy\right)|g(t)|dt = \|f\|_1\|g\|_1\]
    Получаем:
    \[\left|\int_{\R^m}f(x - t)g(t)\right| \le \int_{\R^m}H(x)dx = \|f\|_1\|g\|_1\]
    Тогда \(|f(x - t)g(t)|\) конечна почти всюду \(\Ra (f * g)(x)\) определена почти всюду.
\end{proof}

\begin{theorem}
    \begin{enumerate}
        \item Если \(f \in L_p(\R^m), g \in L_q(\R^m), \frac{1}{p} + \frac{1}{q} = 1\), то \((f * g)\) существует, равномерно непрерывна на \(\R^m\) и \(|(f * g)(x)| \le \|f\|_p\|g\|_q\)
        \item Если \(f \in L_p(\R^m), g\) измерима и ограничена в \(\R^m\), то \((f * g)\) существует, равномерно непрерывна на \(\R^m\) и \(|(f * g)(x)| \le \|f\|_p\|g\|_\infty\), где \(\|g\|_\infty = \sup_{\R^m} |g|\)
    \end{enumerate}
\end{theorem}
\begin{proof}\indent
    \begin{enumerate}
        \item \[\int_{\R^m} |f(x - t)g(t)|dt \le \left( \int_{\R^m}|f(x - t)|^p dt\right)\frac{1}{p} \left( \int_{\R^m}|g(t)|^q dt\right)\frac{1}{q}= \|f\|_p\|g\|_q < \infty\]
        Тогда \(f * g\) определена всюду на \(\R^m\), и справедлива оценка из условия. Докажем равномерную непрерывность:
        \[|(f * g)(x - h) - (f * g)(x)| = \left| \int_{\R^m}(f(x - h - t) - f(x - t))g(t)dt \right| = \]
        \[= |((f_h - f) * g)(x)| \le \|f_h - f\|_p\|g\|_q \ra 0, h \ra 0\]
        Последнее стремление равномерное, т.е. не зависит от \(x\), поэтому утверждение доказано
        \item \[\int_{\R^m} |f(x - t)g(t)|dt \le \|g\|_\infty \|f\|_1 < \infty\]
        Равномерное стремление доказывается аналогично пункту 1.
    \end{enumerate}
\end{proof}

\begin{proposition}
    Пусть \(f, g, h \in L_1(\R^m)\), тогда:
    \begin{enumerate}
        \item \((f * g) = (g * f)\)
        \item \((f * g) * h = f * (g * h)\)
    \end{enumerate}
\end{proposition}
\begin{proof}\indent
    \begin{enumerate}
        \item \[(f * g) = \int_{\R^m} f(x - t)g(t) dt = \left\{\begin{array}{c}
            x - t = y \\
            |J| = 1
        \end{array}\right\} = \int_{\R^m} f(y)g(x - y)dy = (g * f)\]
        \item Следует из теоремы Фубини
    \end{enumerate}
\end{proof}

\begin{definition}
    Последовательность \(\{K_n\}_{n = 1}^\infty\) интегрируемых на \(\R^m\) функций называется аппроксимацией единицей (\(\delta\)-образным семейством), если 
    \begin{enumerate}
        \item \(K_n \ge 0 \forall n\)
        \item \(\int_{\R^m} K_ndx = 1\)
        \item \(\lim_{n \ra \infty} \int_{|x| \ge \delta} K_n(x)dx \ra 0\) для всех \(\delta > 0\).
    \end{enumerate}
\end{definition}

\begin{example}
    Рассмотрим \(\phi \in L_1(\R), \phi \ge 0\) с \(\int_{\R^m}\phi(x)dx = 1\), положим \(K_n(x) = n^m\phi(nx)\). Тогда:
    \[\int_{|x| \ge \delta} = \int_{|x| \ge \delta} n^m\phi(nx)dx = \left\{\begin{array}{c}
        y = nx \\
        |J| = n^m
    \end{array}\right\} = \int_{|y| \ge n\delta}\phi dy = \int_{\R^m}\phi I_{\{|y| \ge n\delta\}} dy \ra 0\]
\end{example}
\hypertarget{lecture3}{}
\begin{theorem}
    Пусть \(\{K_n\}_{n = 1}^\infty\) --- аппроксимация единицы и \(f\) --- ограниченная измеримая функиция на \(\R^n\). Тогда справедливы утверждения:
    \begin{enumerate}
        \item Если \(f\) непрерывна в \(x \in \R^m\), то \(K_n * f(x) \ra f(x)\).
        \item Если \(f\) непрервна в каждой точке компакта \(K \subset \R^m\), то \(K_n * f \rightrightarrows f\) на \(K\)
    \end{enumerate}
\end{theorem}
\begin{proof}
    Из пункта 2 определения \(K_n\) следует, что \(f(x) = \int_{\R^m} f(x)K_n(t)dt\), поэтому \(K_n * f(x) - f(x) = \int_{\R^m} (f(x - t) - f(x))K_ndt\). Зафиксируем \(\epsilon > 0\). Пользуясь равномерной непрерывностью \(f\) на \(K\), найдем такое \(\delta > 0\), что \(\forall x \in K \forall t \in \R^m \left(|t| < \delta \Ra |f(x - t) - f(x)| \le \frac{\epsilon}{2}\right)\). Разобьем последний интеграл следующим образом:
    \[K_n * f(x) - f(x) = \left( \int_{|t| < \delta} + \int_{|t| > \delta} \right)(f(x - t) - f(x))K_n(t)dt = I_1 + I_2\]
    Имеем:
    \[|I_1| \le \int_{|t| < \delta} |f(x - t) - f(x)||K_n(t)|dt \le \frac{\epsilon}{2}\int_{\R^m}K_n(t)dt = \frac{\epsilon}{2}\]
    \[|I_2| \le \int_{|t| \ge \delta} (f(x - t) + |f(x)|)K_n(t)dt \le 2C \int_{|t| \ge \delta} K_n(t)dt\]
    Из пункт 3 опрелеления \(K_n\), найдем \(N = N(\delta)\) такое, что \(\forall n \ge N \left( 2C \int_{|t| \ge \delta} K_n(t)dt \le < \frac{\epsilon}{2} \right)\), тогда при таких \(n\) выполнено \(|K_n * f(x) - f(x)| < \epsilon\), т.е. \(K_n * f \rightrightarrows f\) на \(K\).

    Пункт 1 следует из пункта 2, где в качестве компакта рассматривается точка.
\end{proof}

\begin{problem}
    Если \(f \in L_p(\R^m)\), то \(K_n * f(x) \ra f\) в \(L_p\).
\end{problem}

В периодическом случае свертка определяется аналогично.
\begin{definition}
    Пусть \(f, g\) --- \(2\pi\)-периодичны и измеримы на \(\R\). Тогда:
    \[f * g = \int_{-\pi}^{\pi}f(x - t)g(t)dt\]
\end{definition}

\begin{note}
    Все утверждения для свертки верны и в периодическом случае, если заменить \(\R\) на \([-\pi, \pi]\) и воспользоваться следующим утверждением:
\end{note}

\begin{proposition}
    Если \(f\) интегрируема на \([-\pi, \pi]\), то интеграл \(\int_a^{a + 2\pi} f(t)dt\) не зависит от \(a\).
\end{proposition}
\begin{proof}
    Пусть \(b \in \R\). Тогда \(\forall a \in \R \exists k \in \Z: a + 2\pi k \in [b, b + 2\pi)\), тогда:
    \[\int_a^{a + 2\pi} = \int_{a + 2\pi(k - 1)}^{a + 2\pi k} = \int_{a + 2\pi(k - 1)}^{b} + \int_{b}^{a + 2\pi k} = \int_{b}^{a + 2\pi k} + \int_{a + 2\pi k}^{b + 2\pi} = \int_b^{b + 2 \pi}\]
\end{proof}

\begin{definition}
    Пусть \(p \ge 1\), тогда определим \(L_p(T) = \{f: \R \ra \Cm \text{ --- \(2\pi\)-периодичная измеримая, т.ч. }f \in L_p(-\pi, \pi)\}\). Норма определяется аналогично:
    \[\|f\|_p =\left( \int_{\pi}^{\pi} |f|^p \right)^{\frac{1}{p}}\]
\end{definition}

\begin{definition}
    \(C(T) = \{f: \R \ra \Cm \text{ --- \(2\pi\)-периодическая непрерывная функци }\}\) с \(\|f\| = \sup_{[-\pi, \pi]} |f|\).
\end{definition}

\section{Тригонометриеский ряд Фурье}
\begin{definition}
    Ряд:
    \begin{equation}
        \frac{a_0}{2} + \sum_{k = 1}^\infty a_k\cos kx + b_k\sin kx
    \end{equation}
    Называется тригонометрическим рядом с коэффициентами \(a_k, b_k \in \Cm\).
\end{definition}

По формулам Эйлера: \(\cos kx = \frac{e^{ikx} + e^{-ikx}}{2}, \sin kx = -i\left( \frac{e^{ikx} - e^{-ikx}}{2} \right)\), поэтому частичные суммы можно переписать в виде:
\[S_n(x) = \frac{a_0}{2} + \sum_{k = 1}^n a_k\cos kx + b_k\sin kx = \sum_{k = -n}^n c_ke^{ikx}\]
Где \(c_{\pm k} = \frac{a_k \mp ib_k}{2}, \frac{a_0}{2} = c_0\). 

\begin{definition}
        Ряд:
    \begin{equation}
        \sum_{k = -\infty}^{\infty} c_ke^{ikx}
    \end{equation}
    Называется тригонометрическим рядом в комплексной форме, и его сумма считается как 
    \[\sum_{k = -\infty}^{\infty} c_ke^{ikx} = \lim_{n \ra \infty} \sum_{-n}^n c_ke^{ikx}\]
\end{definition}

\begin{lemma}
    Пусть \(f \in L_1(-\pi, \pi)\) и тригонометрический ряд в форме (2.1) или (2.2) сходится почти всюду к \(f\) и существует \(g \in L_1(-\pi, \pi)\), такая, что \(|S_n(x)| \le g \forall n\) для почти всех \(x\). Тогда:
    \[a_k = \frac{1}{\pi}\int_{-\pi}^\pi f(t)\cos kt, b_k = \frac{1}{\pi}\int_{-\pi}^\pi f(t)\sin kt, c_k = \frac{1}{2\pi}\int_{-\pi}^\pi f(t)e^{-ikx}\]
\end{lemma}
\begin{proof}
    Зафиксируем \(k \in \Z\). По условию, \(S_n(x) e^{-ikx} \ra f(x)e^{-ikx}\) почти всюду. При этом, \(|S_n(x) e^{-ikx}| \le g\). Поэтому, по теореме Лебега о мажорируемой сходимости, имеем: 
    \[\lim_{n \ra \infty} \int_{-\pi}^{\pi}S_n(x)e^{-ikx}dx = \int_{-\pi}^{\pi}f(x)e^{-ikx}dx\]
    Так как \(\int_{-\pi}^{\pi}e^{imx}e^{-ikx}dx = \left\{\begin{array}{l}
        0, k \ne m \\
        2\pi, k = m
    \end{array}\right.\), то 
    \[\int_{-\pi}^{\pi}S_ne^{-ikx}dx = 2\pi c_k, n \ge |k|\]
    Следовательно, \(\int_{-\pi}^{\pi} f(x)e^{-ikx}dx = 2\pi c_k\)
\end{proof}

\begin{definition}
    Пусть \(f \in L_1(T)\). Числа
    \[a_k(f) = \frac{1}{\pi}\int_{-\pi}^\pi f(t)\cos kt, b_k(f) = \frac{1}{\pi}\int_{-\pi}^\pi f(t)\sin kt, c_k(f) = \frac{1}{2\pi}\int_{-\pi}^\pi f(t)e^{-ikx}\]
    Называются коэффициентами Фурье функции \(f\), а ряд
    \[S(f, x) = \frac{a_0(f)}{2} + \sum_{k = 1}^\infty a_k(f)\cos kx + b_k(f)\sin kx = \sum_{k = -\infty}^{\infty} c_k(f)e^{ikx}\]
    Называется тригонометрическим рядом Фурье функции \(f\)
\end{definition}

\begin{note}
    По лемме Римана об осцилляции, \(a_k(f), b_k(f) \ra 0\) при \(k \ra \infty\) и \(c_k(f) \ra 0\) при \(|k| \ra 0\).
\end{note}

\subsection{Поточечная сходимость рядов Фурье}
\begin{definition}
    \(D_n(t) = \frac{1}{2}\sum_{k = -n}^n e^{ikt} = \frac{1}{2} + \sum_{k = 1}^n \cos kt\) называется \(n\)-ым ядром Дирихле.
\end{definition}

\begin{note}
    Функция \(D_n\) непрерывна, \(2\pi\)-периодическая, \(\int_{-\pi}^{\pi} D_n(t)dt = \pi\).
\end{note}

Имеем:
\[S_n(f, x) = \sum_{k = -n}^n\left( \frac{1}{2\pi}\int_{-\pi}^{\pi}f(s)e^{-iks}ds \right)e^{ikx} = \frac{1}{2\pi}\int_{-\pi}^\pi f(s)\sum_{k = -n}^n e^{ik(x - s)}ds = \frac{1}{\pi}\int_{-\pi}^{\pi} f(s)D_n(x - s)ds = \]
\[\frac{1}{\pi}D_n * f(x) = \frac{1}{\pi}f * D_n(x)\]

\[D_n(t) = \frac{1}{2}e^{-nit}\left( 1 + e^{it} + \dots + e^{2nit} \right) = \frac{1}{2}e^{-nit} \frac{e^{2(n + 1)it} - 1}{e^{it} - 1} = \frac{e^{\left( n + \frac{1}{2} \right)it} - e^{-\left( n + \frac{1}{2} \right)it}}{2 \left( e^{\frac{it}{2}} - e^{-\frac{it}{2}} \right)} = \frac{\sin\left( n + \frac{1}{2} \right)t}{2 \sin \frac{t}{2}}\]

\begin{lemma}
    Пусть \(f \in L_1(T)\) и \(0 < \delta < \pi\). Тогда \(\forall x \in \R\) выполнено:
    \[S_n(f, x) = \frac{1}{\pi}\int_{-\delta}^{\delta}\frac{f(x - t)}{t}\sin\left( n + \frac{1}{2} \right)t dt + \epsilon_n(x), \lim_{n \ra \infty} \epsilon_n(x) = 0\]
\end{lemma}
\begin{proof}
    Имеем:
    \[S_n(f, x) = \frac{1}{\pi}\int_{-\pi}^{\pi}f(x - t)\frac{\sin\left( n + \frac{1}{2} \right)t}{2\sin\frac{t}{2}}dt = \]
    \[ = \frac{1}{\pi}\int_{|t| < \delta} \frac{f(x - t)}{t}\sin\left( n + \frac{1}{2} \right)tdt + \frac{1}{\pi}\int_{\delta \le |t| \le \pi} \frac{f(x - t)}{t}\sin\left( n + \frac{1}{2} \right)t + \]
    \[+ \frac{1}{\pi}\int_{-\pi}^\pi f(x - t)\left( \frac{1}{2\sin \frac{t}{2}} - \frac{1}{t} \right)\sin\left( n + \frac{1}{2} \right)t\]
    \(h(t) = f(x - t)\) интегрируема на \((-\pi, \pi)\). Положим \(g(t) = \frac{1}{2\sin \frac{t}{2}} - \frac{1}{t}\). По непрерывности, доопределим \(g(0) = 0\) --- то есть \(g\) непрерывна на \([-\pi, \pi] \Ra\) ограничена.  Аналогично, \(\frac{f(x - t)}{t}\) интегрируема на \(\pi \ge |t| \ge \delta\). Но тогда по лемма Римана об осцилляции, получаем желаемое.
\end{proof}
